\documentclass{article}
\usepackage[english, russian]{babel}
\usepackage{xcolor}
\usepackage{hyperref}
\usepackage{graphicx}
\title{Эссе}
\author{Гривин Никита}
\date{Июнь 2021}

\begin{document}

\maketitle
\section{Введение, описание проблемы}
    \qquad Наш проект заключен в разроботке робота для сбора мусора средних размеров. Данный проект нужен чтобы собирать мусор на неком простарнстве, предположительно улице,далее доставка на некую "базу" для поселдующей утилизации. Сейчас данный проект поможет сделать окружающее нас простарнство чище, что в нашу эпоху немало важно, так как нас окружает огромное число мусора.Возможности решения данной проблемы велики, но все они основываются на робототехнике. Обычно используется робот в котором есть большой контейнер и хват.Хват может представлятся в виде ковша, крана и ручного хвата.Многие государтсва разрабатывают своих роботов уборщиков, чтобы убирать улицы независимо от людей, напрмер, , он предназначен для уборки внега в погоду,чтобы люди не выходили убирать улицы в плохую погоду. \parindent=2em
    
   \par Решаемая мной проблема заключается в нахождение объекта по данным с камеры. По сути, данная пробелма заключается в получении изображения с камеры и обработка его таким образом, чтобы мы смогли узнать , есть ли нужный нам объект в пределах видимости  камеры.Решаемая задача в современное время распростарнена, например, она встречается в робототехнике или в задачах связанным с развитием общего \textcolor{blue}{\href{https://en.wikipedia.org/wiki/Artificial_intelligence}{ИИ}}. Ее можно встретить в любых сферах деятельности связанных с камерами, так как правильно решеная задача и оптимизированное решение позволят оптимально расходывтаь ресурсы. Например, ИИ должно понимать, что перед ним находится как можно быстрее так как,елси автомобильное ИИ не поймет, что идет пешеход - это может привести к аварии.
\section{Текущее состояние области исследования}
    \qquad На сегодяншней день предожено множество способов решение этой задачи от сильных нейросетей общего назначения до простого побитового анализа картинок. Каждое из решений имеет свою область назначения, а также свои плюсы и минусы, которые влияют на работу алгоримтов. В последнее время из-за популярности ИИ стали распространены нейронные сети. Все нейросети делятся на три вида: обучение с учителем, без учителя, и смешанное обучение. В моей сфере деятельности можно выделить 4 вида сетей,  а именно:
    \begin{itemize}
    
        \item \textcolor{blue}{\href{https://en.wikipedia.org/wiki/Perceptron}{перцептрон}}
        \item \textcolor{blue}{\href{https://en.wikipedia.org/wiki/Convolutional_neural_network}{сверточная нейронная сеть}}
        \item \textcolor{blue}{\href{https://en.wikipedia.org/wiki/Radial_basis_function_network}{сеть радиально-базисных функций}}
        \item \textcolor{blue}{\href{ht tps://en.wikipedia.org/wiki/Adaptive_resonance_theory}{адаптивные резонансные сети}}
    \end{itemize}
    \par Данные нейросети явялются только одним из методов решения проблемы, например занменитая \textcolor{blue}{\href{https://en.wikipedia.org/wiki/Boston_Dynamics}{Boston Dynamics}} только в последнее время перешла на нейронные сети для анализа снимков, до этого они использовали жесткую алгоритмизацию действий, которая позволяла работать эффективно в известны ситуциях, но к сожалению не обучаться новым
    
 
\section{Описание методов}
        \qquad Я в решении поставленной перед мной задачи использовал алгоритм, завязанный на подматричном анализе снимка.На проекте мы были вынуждены отказаться от нейронных сетей из-за сложности их реализации и обязательном процессе обучения сетей, которое мы не можем им предоставить. Мы считаем, что мусором на нашем полигоне будет являтся крансый объект для упрощенного поиска.Как например этот объект на полигоне.
    \includegraphics[scale=0.25]{red_obj.png}
    \par Мой алгоритм должен найти красный объект вокург робота и с помощью корректировки скорости остановить перед ним  робота.Это позволит роботу в последующей задачи с помощью лидара подъехать к нему и остановится на некотором расстоянии, чтобы в дальнейшем с помощью хвата забрать мусор. Я предложил решение этой проблемы в виде анализа матрицы снимка с помощью \textcolor{blue}{\href{https://e-maxx.ru/algo/maximum_zero_submatrix    }{алгоритма поиска наибольшей нулевой подматрицы}}, где в матрице нулевыми элементами будут считаться красные элементы, которые в \textcolor{blue}{\href{ https://en.wikipedia.org/wiki/RGB_color_model    }{$RGB$ раскраске}} имеют $R>100; G<10; B<10$.Нам поступает массив данных в виде $RGB$ массива.\newline
        \includegraphics[scale=0.5]{array.png}
        \par Так как в этом массиве нам очень неудобно работать, мы преобразуем его в матрицу (вектор векторов) из $0$ и $1$, где соответсвенно 1 - не удовлетворяющие нашим параметрам пиксели. После этого, с помощью выбранного нами алгоритма мы узнаем какого размера наибольшая нулевая подматрица. Стоит отметить, что при простом переборе мы бы нашли наибольший прямоугольник за $O(n^2m^2)$, что при нашем изображении в $640*480$ очень дорого, а именно, нам придется сделать $9,43*10^9$ операций процессора без учета константы. В используемом алгоритме мы получаем  \textcolor{blue}{\href{https://en.wikipedia.org/wiki/Asymptotic_analysis   }{асимптотику}} равную $O(nm)$ с помощью динамического  программирования. Эта асимптотику в худшем случаем даст 307200, а это очень хороший результат. \parДалее мы сравниваем результаты с наименьшими критериями для отбора случайных аномалий: размер подмарицы не более чем $20pix^2$ мы считаем случайностью (засвет, кусок объекта в кадре и т.д.).
        \par Далее с помощью формулы мы изменяем угловую скорость в зависмости от разности центральной оси изображения и центральной оси объекта на изображении.
    \newline
    Формула:
    \begin{displaymath}
        \omega= \frac{cent_{picture}-cent_{obj}}{cent_{picture}}  * \omega_0
    \end{displaymath}
    \newline Где:\newline
    $cent_{obj}$ -- ценральная ось объекта на изображении\newline
    $cent_{picture}$ -- центральная ось изображения
    \par Мы остановимся когда центры совпадут и их отношение будет равно единице.
    Будем считать, что отношение равно 1 когда разница между двумя осями не составит больше 15 пикселей.
    \newline На этом изображении приведена блок-схема алгоритма.\newline
    \includegraphics[scale=0.4]{image.png}
\section{Результат работы, выводы}
    В результате работы я сделал навык робота в виде ros-сервсиса, в котором реализован поиск объекта по данным с камеры и остановка перед ним, если он найден.\par
    В результате работы я научился работать с Git-репрезиториями, рабоать в системе ROS,а также командной работе.
\section{Список литературы}
    \begin{itemize}
        \item \textcolor{blue}{\href{http://wiki.ros.org/ru}{Работа с сервисами     ROS и     с самим ROS}}
        \item \textcolor{blue}{\href{https://e-maxx.ru/algo/maximum_zero_submatrix    }{Алгоритм поиска наибольшей нулевой подматрицы}}
        \item \textcolor{blue}{\href{https://github.com/Neuromatrix/ROS-camera-vie    w}{Э    то ссылка на рабочий проект в GitHub}}
    \end{itemize}

    
\end{document}
